% MNS project 2010 report
%
% Luís Francisco Seoane Iglesias
% Mirko Dietrich - mirko.dietrich -AT- bccn-berlin -DOT- de
% 

\documentclass[a4paper,12pt,oneside]{article}

\usepackage[utf8x]{inputenc}
\usepackage{amsmath}
\usepackage{cite}
\usepackage{graphicx}
\usepackage{epsfig}
%\usepackage{html}
% Line spacing
\usepackage{setspace}

%\sloppy

% typeset physical units
\newcommand{\unit}[1]{\ensuremath{\, \mathrm{#1}}}

\begin{document}

\begin{titlepage}
\begin{center}
\begin{spacing}{2}
{\huge\bf Spike-timing-dependent plasticity} \\
\end{spacing}
\Large
Project report \\
\vfill
\normalsize
\today \\
\vspace{5em}
Models of Neural Systems WS 09/10 \\
\vspace{1em}
Luís Francisco Seoane Iglesias \\
Mirko Dietrich \\
\vspace{1em}
Bernstein Center for Computational Neuroscience Berlin
\end{center}
\end{titlepage}

\tableofcontents

\newpage

%%%%%%%%%%%%%%%%%%%%%%%%%%%%%%%%%%%%%%%%%%%%%%%%%%%%%%%%%%%%%%%%%%%%%%%%%%%%%%%
\section{Introduction}

While the Hebbian learning rule says that \textit{cells that fire
  together, wire together} the Spike-timing-dependent plasticity
theory states more precisely by describing a relation between synapse
plasticity and the time between pre- and postsynaptic spikes. It finds
when a presynaptic precedes a postsynaptic spike the corresponding
synapse connectivity is strengthened. On the other hand when a
presynaptic follows a postsynaptic action potential the efficacy of
synapse transmission decreases.\cite{Song:2000}

%%%%%%%%%%%%%%%%%%%%%%%%%%%%%%%%%%%%%%%%%%%%%%%%%%%%%%%%%%%%%%%%%%%%%%%%%%%%%%%
\section{Methods}

We modelled a neuron with a number of inhibitory and excitatory
synapses connected to it. The neuron's voltage is calculated using a
modified integrate-and-fire model. Synaptic conductances $g_{in}$ and
$g_{ex}$ change with time:

\[
\tau_m \frac{dV}{dt} = V_{rest} – V + g_{ex}(t)(E_{ex} – V) + g_{in}(t)(E_{in} – V)
\]

A postsynaptic action potential is triggered when the cell's resting
potential reaches a threshold of $-54\unit{mV}$. Afterwards the
voltage is reset to $-60\unit{mV}$.

Whenever a presynaptic spike arrives the synapse conductance is changed:

\[
g_{ex}(t) \rightarrow g_{ex}(t)+\bar{g}_{a}
\]

\[
g_{in}(t) \rightarrow g_{in}(t)+\bar{g}_{in}
\]

If no presynaptic spike occurs the conductance decays exponentially:

\[
\tau_{ex} \frac{dg_{ex}}{dt} = -g_{ex},\hspace{1em} \tau_{in} \frac{dg_{in}}{dt} = -g_{in}
\]

\subsection{Parameters}

The following parameters were used for the simulation:

\begin{align*}
  \tau_m &= 20\unit{ms} \\
  V_{rest} &= -70\unit{mV} \\
  E_{ex} &= 0\unit{mV} \\
  E_{in} &= -70\unit{mV} \\
  \tau_{ex} = \tau_{in} &= 5\unit{ms} \\
  \bar{g}_{in} &= 0.05 \\
  \bar{g}_{max} &= 0.0015
\end{align*}

%%%%%%%%%%%%%%%%%%%%%%%%%%%%%%%%%%%%%%%%%%%%%%%%%%%%%%%%%%%%%%%%%%%%%%%%%%%%%%%
\section{Results}

%%%%%%%%%%%%%%%%%%%%%%%%%%%%%%%%%%%%%%%%%%%%%%%%%%%%%%%%%%%%%%%%%%%%%%%%%%%%%%%
\section{Discussion}

Discuss your results as well as potential problems that you
encountered. Also mention any open questions and potential future projects.

%%%%%%%%%%%%%%%%%%%%%%%%%%%%%%%%%%%%%%%%%%%%%%%%%%%%%%%%%%%%%%%%%%%%%%%%%%%%%%%
\section{Contributions}
Describe who did what (programming, writing of the report, figures,...)

%%%%%%%%%%%%%%%%%%%%%%%%%%%%%%%%%%%%%%%%%%%%%%%%%%%%%%%%%%%%%%%%%%%%%%%%%%%%%%%
\newpage
\bibliographystyle{plain}
\bibliography{report}

\end{document}
